% Created 2015-01-16 Fri 22:40
\documentclass[11pt]{article}
\usepackage[utf8]{inputenc}
\usepackage[T1]{fontenc}
\usepackage{fixltx2e}
\usepackage{graphicx}
\usepackage{longtable}
\usepackage{float}
\usepackage{wrapfig}
\usepackage{rotating}
\usepackage[normalem]{ulem}
\usepackage{amsmath}
\usepackage{textcomp}
\usepackage{marvosym}
\usepackage{wasysym}
\usepackage{amssymb}
\usepackage{hyperref}
\tolerance=1000
\author{Diego Berrocal}
\date{\today}
\title{ClassNotes}
\hypersetup{
  pdfkeywords={},
  pdfsubject={},
  pdfcreator={Emacs 24.4.1 (Org mode 8.2.10)}}
\begin{document}

\maketitle
\tableofcontents

Ejercicios Sugeridos: Libro Introduction to Quantum Mechanics Griffiths 2nd Edition
4.18, 4.28, 4.29, 4.30, 4.31, 4.34, 4.35

\section{Axiomas de la Mecánica cuántica}
\label{sec-1}
\subsection{Axioma 1}
\label{sec-1-1}
Todo sistema físico es completamente descrito con ua fuente de onda $\Psi$ en un espacio de Hilbert.
\subsection{Axioma 2}
\label{sec-1-2}
Para todo observable físico existe un operador hermitiano en H.
$$ x -> \vec{x} $$
$$ p -> \delta_x $$
$$ ? -> \vec{s} $$
\subsection{Axioma 3}
\label{sec-1-3}
Los únicos resultados posibles de la medida de un observable es un
valor del operador correspondiente.
\subsection{Axioma 4}
\label{sec-1-4}
Si un sistema está en un estado $\Psi$ entonces la probabilidad de
obtener valores entre $\lambda$$_{\text{1}}$ y $\lambda$$_{\text{2}}$ es

$$ P(\lambda_1, \lambda_2) = || \Psi ||^2 = || (E(\lambda_2) - E(\lambda_1) )\Psi ||^2$$

E($\lambda$) es la resolución de la indentidad del observable A (\textbf{WHAT}
libros de análisis funcional, teoría de la medida, Libro de análisis funcional es Bresis)
\subsection{Axioma 5}
\label{sec-1-5}
En todo sistema existe un operador hamiltoniano H que determina la evolución temporal.

$$ \Psi (t_1) -> \Psi(t_2) $$
$$ t_1 < t_2 $$
$$ H\Psi(x,t) = i\hbar \frac{\partial\Psi(x,t)}{\partial t} $$
\section{Resultados de la clase anterior}
\label{sec-2}
Aquí $\mu$ es el momento angular o tiene que ver con ello, checka el Gridffiths
Se tiene:
$$ J_\pm = J_1 \pm iJ_2 $$
$$ [J_3,J_\pm] = \pm \hbar J_\pm $$
$$ [J^2, J_\pm] = 0 $$

$$ J^2 = J_\pm J_\mp + J_3^2 \mp\hbar J_3 $$

De donde:

$$ J_3 (J_\pm f) = (\mu \pm \hbar)(J_\pm f) $$

\begin{equation}
\label{baseeq}
\begin{split}
J_3 f & = \mu f   \\
J^2 f & = \lambda f ---> (\lambda > 0) \\
\mu,\lambda \in \mathbb{R}
\end{split}
\end{equation}


$$ -\sqrt\lambda \leq \mu \leq \sqrt\lambda $$
$$ <J^2> \geq <J_3> -> \lambda \geq \mu^2  $$

Hasta ahora se tiene que el operador J\_- o J + sube de energía a los
$\mu$ en un $\hbar$. Como se ve en la siguiente figura. \textbf{Insert Image Here}

Por el teorema del supremo en este espacio acotado de energías, se
tendrá que hay un $\mu$$_{\text{máx}}$ y este será igual a

$$ \mu_máx = l\hbar $$

Y se tendrá que en este $\mu$$_{\text{máx}}$ si se aplicara de nuevo el operador de
creación J + se tendría

$$ J + f_t = 0 $$

Lo mismo pasaría del otro extremo

$$ J - f_b = 0 $$

Al aplicar J$^{\text{2}}$ (la versión con el +) a f$_{\text{t}}$

\begin{equation}
\label{eq1}
\begin{split}
  J^2 f_t & = (J_- J + + J_3^2 + \hbar J_3) f_t \\
          & = J_-(J + f) + (J_3^2 + \hbar J_3)f_t \\
          & = [(l\hbar)^2 + \hbar l\hbar] f_t \\
          & = \hbar^2 (l) (l+1)f_t \\

\end{split}
\end{equation}

$$ l \in \mathbb{R} $$

$$ \lambda = \hbar^2 (l) (l+1) $$

Similarmente aplicamos la versioón con el - a f$_{\text{b}}$

\begin{equation}
\label{eq2}
\begin{split}
  J^2 f_t & = (J_- J + + J_3^2 - \hbar J_3) f_t\\
          & = J_-(J + f) + (J_3^2 - \hbar J_3)f_t \\
          & = [(l\hbar)^2 - \hbar l\hbar] f_t \\
  & = \hbar^2 (l) (l-1)f_t \\

\end{split}
# l \in \mathbb{R}\\

# \lambda = \hbar^2 (l) (l+1)
\end{equation}

Por lo tanto podemos tener que.
$$J_3(J + f ) = (\mu + \hbar)(J + f) $$
$$ \J^2(J + f) = \lambda(J + f) $$

$$ \hbar^2 l (l + 1) = \hbar^2 \vec{ l } (\vec{l} - 1) $$
$$ \vec{l} = l + 1 -> \vec{l} > l, ESTO ESTA MAL $$
$$ \vec{l} = -l , l \in \mathbb{R}$$


$$ l = -l + N, \: N=0,1,2,...$$

$$ l = \frac{N}{2} $$


Para cada l existen iguales valores propios de J$_{\text{3}}$ que representamos
con m de la siguiente manera

$$ J_3 f_{lm} = m\hbar f_{lm} $$
$$ J^2 f_{lm} = \hbar^2 l(l+1) f_{lm} $$
\subsection{Ejemplo con Spin 1/2}
\label{sec-2-1}

Cuando consideramos el espín de 1/2 se considera como partícula de
estudio al electrón, protón, neutrón, quark, leptons, etc. Lo que se
hará ahroa es un ejemplo en el que se suman dos espines 1/2 es decir
el acoplamiento de dos partículas con spin 1/2. Todos las partículas
que tienen espín 1/2 son fermiones, las que tienen espín entero son
bosones como el fotón y el gravitrón.

Cuando se etudia teoría de momento angular siempre se tienen 3
operadores, y en este caso como es un caso específico de Spín,
usaremos S$_{\text{x}}$, S y, S z, y también consideraremos a un S$^{\text{2}}$. La relación
de conmutación que consideraremos será la siguiente:

$$ [S_x,S_y] = i\hbar S_z $$
$$ [S_y,S_z] = i\hbar S_x $$
$$ [S_z,S_x] = i\hbar S_y $$

Se seleccionan 2 operadores del Conjunto de operadores Momento angular
\{S$_{\text{x}}$, S y, S$_{\text{z}}$, S$^{\text{2}}$\}

Este submonjunto \{S$_{\text{z}}$, S$^{\text{2}}$\} se denomina un cConjunto completo de
operadores conmutantes entre sí, podría haberse tomado cualquier
componente del conjunto principal pero por convención se utilizará el
S$_{\text{z}}$.

$$ S^2 |sm> = \hbar^2 s(s+1) |sm>  s= \frac{1}{2} $$
$$ S_z |sm> = \hbar m |sm> $$

s: número cuántico del espín

m: número cuántico de proyección del espín a lo alrgo del eje Z
\subsubsection{Ejercicio}
\label{sec-2-1-1}

$$ S_\pm = S_x \pm iS y $$

Verificar que se cumpla

$$ S_\pm |sm> = \hbar \sqrt{s(s+1) - m(m+1)} |s(m\pm1)> $$
\subsubsection{Ejemplos}
\label{sec-2-1-2}
Sustituir S=½

\begin{equation}
\begin{split}
S^2 |sm> & = \hbar^2 (\frac{1}{2})(\frac{1}{2} + 1) |sm> \\
         & = \hbar^2(\frac{3}{4}) |sm>
\end{split}
\end{equation}

Los valores posibles de m son -½ y ½

2s+1 = 2(½) + 1 = 2

2s+1: Número de valores diferentes posibles

$$ | \uparrow >  = | ½ , ½ >$$
$$ | \downarrow >  = | ½ ,- ½ >$$

Existen 2 vectores linealmente independientes de tal modo que
$$ {|\uparrow>, |\downarrow>} $$

constituyen una base.

UN estado general se expresa mediante una combinación lineal (con coeficientes cojmplejos)

$$ | \Psi > = a |\uparrow > + b | \downarrow> $$

Con $a,b \in \mathbb{C}$

Notación:
$$ |\uparrow> = (1 0) | vertical = \chi + $$
$$ |\downarrow> = (0 1) | vertical = \chi - $$

$$ \Psi_{½½} = | \frac{1}{2}, \frac{1}{2}> = | \uparrow> = (1 0) vertical = \chi + $$
$$ \Psi_{½-½} = | \frac{1}{2}, -\frac{1}{2}> = | \chinarro> = (0 1) vertical = \chi - $$

Un estado general es $\chi$.
$$ \chi = a\chi + = b\chi - $$
\subsubsection{Suma de momento angular}
\label{sec-2-1-3}
$$ S^2, S_z$$
Sabemos que

\left\{
\begin{array}
S^2 \chi +  = \frac{3}{4} \hbar^2 \chi +\\
S^2 \chi -  = \frac{3}{4} \hbar^2 \chi -\\
\end{array}

Cuado se usa funciones de onda con columnas de 2 componentes se dice
que se está usando la representación espinorial.

$$ \chi = a\chi + = b\chi - $$

\begin{pmatrix}
a\\b
\end{pmatrix}
= a
\begin{pmatrix}
1\\0
\end{pmatrix}
+ b
\begin{pmatrix}
0\\1
\end{pmatrix}

Calculamos la representación del operador S$^{\text{2}}$

S$^{\text{2}}$ =
\begin{pmatrix}
c & d\\e & f
\end{pmatrix}

$$ c,d,e,f \in \mathbb{C} $$

Aplicando  S$^{\text{2}}$ en $\chi$ +



Similarmente aplicamos S$^{\text{2}}$ en $\chi$ = \begin{pmatrix}0\\1\end{pmatrix}


Bransden
\begin{pmatrix}
c & d\\e & f
\end{pmatrix}

\begin{pmatrix}
0\\1
\end{pmatrix}
= $\hbar$$^{\text{2}}$ \frac{3}{4}
\begin{pmatrix}
0\\1
\end{pmatrix}




\textbf{Check out this book Devanathan, Angular MOmentum in Quantum Mechanics}
% Emacs 24.4.1 (Org mode 8.2.10)
\end{document}