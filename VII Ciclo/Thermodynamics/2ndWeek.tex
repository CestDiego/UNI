% Created 2014-09-24 mié 22:06
\documentclass[11pt]{article}
\usepackage[utf8]{inputenc}
\usepackage[T1]{fontenc}
\usepackage{fixltx2e}
\usepackage{graphicx}
\usepackage{longtable}
\usepackage{float}
\usepackage{wrapfig}
\usepackage{rotating}
\usepackage[normalem]{ulem}
\usepackage{amsmath}
\usepackage{textcomp}
\usepackage{marvosym}
\usepackage{wasysym}
\usepackage{amssymb}
\usepackage{hyperref}
\tolerance=1000
\usepackage[english]{babel}
\author{Diego Berrocal}
\date{\today}
\title{2ndWeek}
\hypersetup{
  pdfkeywords={},
  pdfsubject={},
  pdfcreator={Emacs 24.3.1 (Org mode 8.2.7c)}}
\begin{document}

\maketitle
\tableofcontents

\section{4 pasos para resolver problemas de estadística.}
\label{sec-1}
Vamos a aplicar las cosas que debemos hacer en los problemas.
\begin{itemize}
\item Especificación del estado del sistema

\item Debemos tener un Conjunto Estadístico
\end{itemize}
Aquí se llama un Ensamble. En física estadística no se considera un
sólo sistema sino una colección de sistemas idénticos.
\begin{itemize}
\item Necesitamos postulados básicos o también llamados (probabilidades a priori)
\item Cálculo del Problema
\end{itemize}

\subsection{Especificación del estado del Sistema}
\label{sec-1-1}
\begin{itemize}
\item Ejemplo 1
\end{itemize}
Partícula con Spin ½
\begin{center}
\begin{tabular}{ll}
½ , ½ > , & ½ , -½ >\\
\end{tabular}
\end{center}
Estos son los spines que son respectivamente el spin hacia arriba o el de hacia abajo.
\begin{itemize}
\item Ejemplo 2
\end{itemize}
N Partículas de spin ½


\begin{center}
\begin{tabular}{l}
m$_{\text{1}}$, m$_{\text{2}}$, m$_{\text{3}}$, \ldots{} , m$_{\text{n}}$ >\\
\end{tabular}
\end{center}


Cada uno de estas partículas puede ser de spin ½ o -½ para N grande,
por ejemplo N= N a = 6.02x10^2^3
\begin{itemize}
\item Ejemplo 3
\end{itemize}
Oscilador Armónico 1-dim
\begin{center}
\begin{tabular}{l}
n> n = 0,1,2\ldots{}..\\
\end{tabular}
\end{center}
En = hw(n+½)
\begin{itemize}
\item Ejemplo 4
\end{itemize}
N Osciladores armónicos
\begin{center}
\begin{tabular}{l}
n1, n2, \ldots{}. ,n N >\\
\end{tabular}
\end{center}
ni = 0,1,\ldots{}.
V i $\epsilon$ \{1,2,\ldots{}.,N\}
\begin{itemize}
\item Ejemplo 5
\end{itemize}
En un cubo las 3 dimensiones de este son Lz,Ly,Lx y entonces la
función de onda para una partícula en este cubo sería:

$\Psi$(x,y,z) = Sen(n$\pi$ x / Lx) Sen(n$\pi$ y / Ly) Sen(n$\pi$ z / Lz)

\subsection{Conjunto Estadístico}
\label{sec-1-2}

En general, sea la descripción cuántica o la descripción clásica para
especificar el estado de un sistema nosotros vamos a enumerar todos
los estados posibles para poder obtener ese espacio muestral.

En el caso de los estados cuánticos se niumerará usando el índice r, con


r=1,2,3\ldots{}.

Nosotros usaremos la notificación de una función de onda con su número
cuántico r correspondiente, esto es que todosl os sistemas posibles se
dan para el conjunto de la variación de toda la función de onda con el
número cuántico

Ahora en vez de considerar un único sistema con varios estados, vamos
a considerar muchos sistemas idénticos, pero no implica que estarán en
el mismo estado. Esto es , un Ensamble (Conjunto en francés)

\subsubsection{Ejemplo}
\label{sec-1-2-1}
Sean 3 partículas de spin ½ en un campo magnético con intensidad H


Sea $\mu$ = $\gamma$ s

E = - $\mu$ . H

\begin{center}
\begin{tabular}{rlll}
r & m1,m2,m3 & Total de Momento Magnetico & Total de Energía\\
\hline
1 & + + + & 3$\mu$ & -3$\mu$ H\\
2 & + + - & $\mu$ & -$\mu$ H\\
3 & + - + & $\mu$ & -$\mu$ H\\
4 & - + + & $\mu$ & -$\mu$ H\\
5 & + - - & -$\mu$ & $\mu$ H\\
6 & - + - & -$\mu$ & $\mu$ H\\
7 & - - + & -$\mu$ & $\mu$ H\\
8 & - - - & -3$\mu$ & 3$\mu$ H\\
\end{tabular}
\end{center}

 Típicamente se conoce información parcial del sistema, es decir
conocemos un parámetro macroscópico que no necesariamente nos daría
información detallada.

\subsubsection{Ejercicio}
\label{sec-1-2-2}
Supondremos que el sistema de 3 spines especificado anteriormente de ½
está aislado, lo que significa que no intercambia energía con el
exterior. En este caso son sólo hay un sistema de 3 espines, pero en
TermoEstadística lo que haremos es considerar muchos sistemas
idénticos de 3 spines. Cada uno con una configuración no
necesariamente igual a los demás pero siempre considerando que la
energía se conserva.

\subsubsection{Consideremos}
\label{sec-1-2-3}
Conocemos el parámetro macroscócpo de Energía Total E = - $\mu$ H.

Esto nos deja con los posibles estados cuánticos r=2,3,4

La sub-colección de estados compatibles con la energía total E = - $\mu$
H se llama el \textbf{Conjunto de estados compatibles}

Esto quiere decir que el sistema se encuentra en uno de estos estados,
o ya más generalmente se podría decir que está en una combinación
lineal de todos estos, ya que igual haría que la energía total se
conserve.

\{(+ + -), (+ - +), (- + +)\}

\subsection{Postulados Básicos}
\label{sec-1-3}
\begin{itemize}
\item En general existe un número grande de estados accesibles
\item Estar en equilibrio implica que la probabilidad de estar en un
estado accesible particular NO depende del tiempo
\item Para sistemas \textbf{aislados en equilibrio} TODOS los parámetros son
independientes del tiempo
\item Sólo se sabe que el sistema tiene que estar en uno de los estados
accesibles
\item Las leyes de la mecánica cuántica no permiten calcular las
probabilidades de cada estado accesible
\item Es lógico suponer que cuando se tiene sistemas aislados en
equilibrio las probabilidades de cada estado accesible son iguales
\end{itemize}

(Hipótesis  + veríficada muchas veces) --> Postulado

\begin{itemize}
\item La distribución de probabilidad de los estados accesibles es Uniforme
\item Una vez que un sistema adquiere el equilibrio permanecerá en tal hasta
que un agente externo lo saque del equilibrio (Inercia)
\end{itemize}
\subsubsection{Ejemplo}
\label{sec-1-3-1}

( + + -) ,  ( + - +), ( - + + )

Cualquiera de estos tiene una probabilidad de 1/3
\subsubsection{Ejemplo 2}
\label{sec-1-3-2}

Oscilador armónico unidimensional ( Aproximación clásica)

E = p²/2m + ½Kx²


Supongamos que la energía toma valores entre E y E+$\delta$ E

Si llevamos esto al espacio fásico de P VS X entonces tendríamos la
imágen de una elipse para cada valor de energía, entonces para E y E +
$\delta$ E habrán elipses para valores entre estos dos topes se tiene
una distribución de la posición. Cuya probabilidad se ve dada por la
relación entre los estados accesibles que se tenga entre las dos
elipses dibujadas y el 'numero total de elipses que pueden caber en el
area que se forma entre las dos elipses'

Pero obviamente si se tiene que todos los posibles espacios que se
puedan llenar en el area formada por las dos elipses son iguales,
entonces la probabilidad por definición para cada uno de todos estos
estados de posición es la misma.


\textbf{Sin embargo}, si las probabilidades de los estados cuánticos son
 diferentes entre sí, decimos que el sistema no está en equilibrio.
 Eventualemete el sistema va a tender hacia la condición de
 equilibrio. (Es decir las probabilidades cambian con el tiempo)
 Eventualmente el sistema va a tender hacia la condición de equilibrio

\begin{center}
\begin{tabular}{lll}
Sistema que no está en equilibrio & -----> & Sistema que está en equilibrio\\
Aislado(Cerrado) & 1 seg & \\
 & 1 siglo & \\
 & Se le llama & \\
 & Tiempo de & \\
 & Relajación & \\
\end{tabular}
\end{center}



Como ejemplo de proceso de conversión al equilibrio se tiene este
ejemplo, Se teiene un recibpiente dividido en dos, en una parte está
lleno con moléculas de cierto tamaño y en la otra mitad está vacío, lo
que se hace como experiencia es separar la división del recipiente,
pasa de un estado en equilibrio en ambas partes a un sistema de no
equilibrio porque ya entran en contacto ambas partes de diferente
densidad de moléculas, lo que pasará a continuación es que el sistema
nuevo comenzará a moverse y distribuyendo su energía dentro de toda la
caja.

Existen Sistemas con estados de equilibrio meta-estable
\subsubsection{Bibliografía recomendada}
\label{sec-1-3-3}
\begin{itemize}
\item Teorema H - Apéndice A12 - Reif : Fundamentos de Termodinámica Estadística
\item \textbf{Tolman - Principles of Statistical Mechanics (Caracterisación de los estados Microscópicos)} Very important and hard
\item Callen - Thermodynamics: an introduction to Thermostatistics (2nd edition) (Macroscopía)
\item 
\item EXTRA: Fórmulas matemáticas - Schaum Series Apiegel
\item Variables aleatorias HSU,
\end{itemize}
\subsection{Cálculo de Probabilidad}
\label{sec-1-4}
Considerar un conjunto de sistemas que están caracterizados por terner
la energía entre E y E + $\delta$ E

\begin{itemize}
\item $\Omega$(E) Es el número total de estados accesibles
$\Omega$(E,y$_{\text{k}}$)  k = 1,2,3,\ldots{}..
\end{itemize}
donde se considera un parámetro macroscópico y que toma valores y$_{\text{1}}$,y$_{\text{2}}$,\ldots{}. (ejemplos de y como presión, momento magnético, etc)

$\Omega$ (E,y$_{\text{k}}$) = número de estados accesibles con valor específico y$_{\text{k}}$ de parámetro y 0

$\Omega$(E) > $\Omega$(E, y$_{\text{k}}$) V k

P(y$_{\text{k}}$) = $\Omega$(E, y$_{\text{k}}$) / $\Omega$(E)


Ejemplo de 3 partículas con espín ½

(+ + -), (+ - +), (- + +)


Consideremos la segunda partícula con espín ½. Tenemos que la segunda partícula tiene los siguientes estados

\begin{itemize}
\item + En el primer caso
\item - En el segundo caso
\item + En el tercer caso

Sea P (Segúnda partícula tenga m$_{\text{s}}$ = ½)
\end{itemize}

y ---> Proyección del espín de la segunda partícula

y$_{\text{1}}$ = ½     $\Omega$ (E,½) = 2
y$_{\text{2}}$ = -½    $\Omega$ (E)   = 3
\subsection{Comportamiento de la densidad de estados}
\label{sec-1-5}

Sistema macroscópico es igual a muchos grados de libertad,se dividirá
la energía E en intervalos pequeños iguales a $\delta$ E, con $\delta$ E
<< E. y a cada $\delta$ E es una medida de precisión al medir E, esto es
la escala mínima que nos permite tener el instrumento de medición.

Aunque $\delta$ E sea pequeño (macroscópicamente) El intervalo cubierto
por $\delta$ E contiene muchos estados posibles (microscópicamente). Ojo
que no es igual al número de estados accesibles. Tan sólo son estados
posibles.

Luego de esto podemos ver que $\Omega$(E) es el número de estados cuya
energía está entre E y E + $\delta$ E.


$\Omega$(E) = W(E)$\delta$ E

Esto quiere decir que $\Omega$(E) va a ser
proporcional a $\delta$ E, a la nueva cantidad que vemos aquí (W(E) la
llamaremos densidad de estados, además W(E) \textbf{no} depende de $\delta$ E

Se hará luego un estimado de $\Omega$ (E) en función de f (grados de
libertad), por lo tanto luego se tiene.

\textbf{f} números de estados cuánticos para especificar cada estado posible.

\textbf{$\Phi$(E)} número de estados cuánticos posibles con energía menor que
E(función creciente)

Cuando uno quiera calcular el $\Omega$(E) sería como calcular el total
de estados dentro de un cascarón (WTF?!)

$\Phi$$_{\text{1}}$ $\alpha$ $\epsilon$$^{\alpha}$   con $\alpha$ $\approx$ 1

$\epsilon$ es la contribución a la energía de un grado de libertad, que
es proporcional a la razón entre la Energía total y el número de
estados cuánticos:

$\epsilon$ $\alpha$ E/f => $\Phi$(E) $\approx$ [$\Phi$$_{\text{1}}$(E)]$^{\text{f}}$ \textbf{Check Tolman}

f \textasciitilde{} 10$^{\text{23}}$  N$_{\text{A}}$ = 6.023 x 10$^{\text{23}}$

$\Omega$(E) = $\Phi$(E+$\delta$ E) - $\Phi$(E)

$\Omega$(E) $\approx$ $\Delta$ Phi /$\Delta$ E $\delta$ E

$\Omega$(E) $\approx$ f $\Phi$$_{\text{1}}^{\text{f}}$-1 $\Delta$ Phi$_{\text{1}}$ /$\Delta$ $\epsilon$ (1/f) $\delta$ E

$\Omega$(E) $\approx$ $\Phi$$_{\text{1}}^{\text{f}}$-1 ($\Delta$ $\Phi$$_{\text{1}}$ /$\Delta$ $\epsilon$) $\delta$ E

Se iene que $\Omega$(E) y W(E) crecen extremadamente rápido con E.

ln($\Omega$) = (f-1) ln $\Phi$$_{\text{1}}$ + ln($\Delta$$\Phi$$_{\text{1}}$ /$\epsilon$   $\delta$ E)

\textbf{Se argumenta que $\Delta$ $\Phi$$_{\text{1}}$ /$\epsilon$ $\delta$ E $\approx$ 1} por ende
 este término en el logaritmo de éste término se desprecia y como f >> 1 entonces queda

ln $\Omega$ = f ln $\Phi$$_{\text{1}}$

$\Omega$(E,f) $\approx$ $\Phi$$_{\text{1}}^{\text{f}}$ $\alpha$ E$^{\text{f}}$

\subsection{Gas ideal en el límite clásico}
\label{sec-1-6}

N moléculas idénticas en un volumen V

E = K + U + E$_{\text{int}}$

K: Energía Cinética

U: Energía potencial entre las moléculas

E$_{\text{int}}$: Energía de rotación y/o vibración de los átomos que constituyen las moléculas

$K = \frac{1}{1m} \sum\limits_{i=1}^N P_i^2 = K(P_1,P_2,....,P_N)$


U(r$_{\text{1}}$,r$_{\text{2}}$,\ldots{}.,r$_{\text{N}}$) Depende de las separaciones relativas entre las
moléculas que interactúan.

\begin{itemize}
\item Si las moléculas son monoatómicas => la energía interna E$_{\text{int}}$ = 0
\end{itemize}
/si las moléculas son poliatómicas se debe considerar E$_{\text{int}}$ $\neq$ 0

E$_{\text{int}}$(Q$_{\text{1}}$,Q$_{\text{2}}$, \ldots{}, Q$_{\text{M}}$, P$_{\text{1}}$, \ldots{} , P$_{\text{M}}$)

nos restringimos al caso simple de moléculas monoatómicas => E$_{\text{int}}$ =0
(Modelo ideal monoatómico) Esta aproximación es útil cuando la
concentración $\frac{N}{V}$ es pequeña.

Cálculo de $\Omega$(E) , se supone que E >> E$_{\text{fundamental}}$

$\Omega$(E) mide el número de estados, es decir , las celdas elementales
en el espaciond e fases clásico entre E y E + $\delta$ E

$$ \Omega(E) \alpha \int_E^{E+\delta E} ... \int\mathrm{d}^3 r_1 ...
\mathrm{d}^3 r_N \mathrm{d}^3 p_1 ... \mathrm{d}^3 p_N \mathrm{d} Q_1
... \mathrm{d} Q_N \mathrm{d} P_1 ... \mathrm{d} P_M = V^N \chi(E) $$

Se tiene
$$\mathrm{d}^3 r_i = \mathrm{d}x_i \mathrm{d} y_i \mathrm{d} z_i \forall i = 1,..,N $$
$$\mathrm{d}^3 p_i = \mathrm{d}P_xi \mathrm{d} P_yi \mathrm{d} P_zi \forall i = 1,..,N $$

E = K (no depende de r$_{\text{i}}$)

$$ \int \mathrm{d}^3 \vec{r_i} = V    \forall i = 1,..,N$$

$$ \chi(E) = \int_E^{E+\delta E} ... \int \mathrm{d}^3 p_1 ...
\mathrm{d}^3 p_N \mathrm{d} Q_1 ... \mathrm{d} Q_N \mathrm{d} P_1 ...
\mathrm{d} P_M $$

$$ [(2mE)^{\frac{1}{2}} ]^2 = \sum\limits_{i=1}^N \sum\limits_{\alpha = 1}^3 P_{i\alpha}² $$

$$ R =(2mE)^{\frac{1}{2}}  $$

$$ P_i^2 = P_1i^2 + P_2i^2 + P_3i^2 $$

$$ \Phi(E) = R^f = (2mE)^{\frac{f}{2}} $$


Primero se calcula el $\Phi$, luego tomando la derivada se calculará el $\Omega$.

$$ => \Omega(E)\ \alpha\ E^{\frac{f}{2} -1}\ \alpha\ E^{\frac{3N}{2} -1 } $$

$$ \Omega(E) = B V^N E^{\frac{3N}{2}} $$

$$ B \in \mathbb{R} $$


\subsection{Interacción Térmica}
\label{sec-1-7}
Sistema A 0 constituido por 2 partes A y A', Cada uno contiene sus
parámetros macroscópicos como lo so Volumen V, Número de partículas N.

Sólo está permitido el intercambio de enrgía pero no de matería o
partículas, por eso este sistema se denomina interacción térmica.

Microscópicamnete esto significa lo siguiente:

A (Antes de la interacción térmica) tendrá una cierta separación entre
sus niveles de energía, y al final de la interacción térmica se tendrá
la misma distribución de niveles de energía cuánticos, ya que las
condiciones macroscópicas no cambian para ninguno.

Sin embargo la distribución de los sistemas idénticos en los niveles
de energía variarán de acuerdo a como se tenga que hacer las
operaciones.


\textbf{insert image here}

Se tiene que lo que pierda o gane A en esta interacción es decir: Q, debe compensarse con lo que pierda o gane A' (Q')

Queda entonces:

$$ Q + Q' = 0 $$


\subsection{Interacción Mecánica}
\label{sec-1-8}

Existe cambio de energía entre los dos subsistemas A y A' con cambio
de parámetros macroscópicos y además no se permite interacción térmica.

Cambia la separación entre niveles de energía y también cambia la
degeneración de los niveles de energía. Antes cuando había interacción
térmica sólo cambiaba la población de los niveles de energía pero
manteniendo los niveles de energía constantes, ahora sin embargo lo
que cambian son los niveles de energía que fueron afectados por la
interacción mecánica, esto causa una diferencia de la Energía que no
se debe por el ordenamiento sino por la energía global. Esta
diferencia de energía por tanto ya no se llamará Q como en el caso
térmico, sino que ahora se llamará W que es alusivo al Trabajo que se
uso para este tipo de interacción
% Emacs 24.3.1 (Org mode 8.2.7c)
\end{document}